\documentclass[11pt,a4paper]{article}

% ---------- Packages ----------
\usepackage[utf8]{inputenc}
\usepackage[T1]{fontenc}
\usepackage{lmodern}
\usepackage{geometry}
\usepackage{float}
\usepackage{placeins}
\usepackage{caption}
\geometry{margin=1in}
\usepackage{setspace}
\onehalfspacing
\usepackage{graphicx}
\usepackage{hyperref}
\usepackage{csquotes}
\usepackage{enumerate}
\usepackage{xcolor}
\usepackage[style=authoryear,maxcitenames=2,backend=biber]{biblatex}

% ---------- Bibliography file ----------
\addbibresource{references.bib}

% ---------- Title ----------
\title{\textbf{The Rainy-Day Fund: Decentralized Parametric Insurance for Smallholder Farmers in Kenya}}
\author{
    Group A: Ellena, Sabina, Noah, Vincent\\[0.5em]
    \textit{University of Basel, Blockchain Challenge 25}
}
\date{\today}

% ---------- Section formatting and ToC ----------
\usepackage{titlesec}
\usepackage{tocloft}
\setcounter{secnumdepth}{3}
\setcounter{tocdepth}{2}
\titleformat{\section}
  {\Large\bfseries}{\thesection}{1em}{}
\titlespacing*{\section}{0pt}{2.0ex plus .2ex minus .2ex}{1.0ex plus .2ex}
\titleformat{\subsection}
  {\large\bfseries}{\thesubsection}{1em}{}
\titlespacing*{\subsection}{0pt}{1.5ex plus .2ex minus .2ex}{0.8ex plus .2ex}
\renewcommand{\cftsecfont}{\normalfont}
\renewcommand{\cftsubsecfont}{\normalfont}
\renewcommand{\cftsecpagefont}{\normalfont}
\setlength{\cftbeforesecskip}{4pt}

\begin{document}
    \maketitle

    \begin{abstract}
        Smallholder farmers in Kenya face serious challenges due to climate variability, as more than 98\% of agriculture depends on rain-fed systems and much of the country’s arable land lies in arid and semi-arid regions (ASALs).
        Frequent droughts, unpredictable rainfall, and extreme weather events have reduced crop yields, threatened food security, and put rural livelihoods at risk.
        Traditional risk mitigation strategies, including crop diversification and off-farm income, remain insufficient, while irrigation development is limited due to infrastructural and environmental constraints.
        Weather index insurance (WII) has emerged as a practical solution because it provides affordable, quick, and reliable payouts based on objective weather data, while reducing administrative costs and moral hazard.
        Building on this idea, the Rainy-Day Fund introduces a decentralized, blockchain-based parametric insurance system.
        By using smart contracts and mobile payment platforms, it can deliver timely payouts to farmers, strengthen their resilience, and help them reinvest in their farms.
        This approach not only protects farmers from the financial shocks of extreme weather but also offers a transparent and efficient way to support sustainable agricultural livelihoods.
    \end{abstract}

    \noindent\textbf{Keywords:} Smallholder farmers; Kenya; climate change; weather index insurance; parametric insurance; blockchain; agricultural resilience; arid and semi-arid lands.

    % Table of Contents
    \tableofcontents
    \bigskip

    % 1 Introduction and Motivation
    \section{Introduction and Motivation}\label{sec:intro}
    The goal of the Rainy Day Fund is to design a decentralized parametric insurance solution that can provide smallholder farmers with affordable, fast, and trustworthy protection against climate-related risks.
    By leveraging blockchain technology, the project seeks to overcome mistrust, high administrative costs, and inefficiencies that have limited adoption of weather index insurance.
    The motivation lies in addressing urgent climate vulnerabilities, reducing poverty traps, and creating scalable financial safety nets in one of the world’s most underserved insurance markets.

    Agriculture is the backbone of Kenya’s economy, contributing approximately 21.3\% to the country’s GDP in 2024~\parencite{WorldBank2024}.
    It is the largest employer in the country, providing livelihoods for over 40\% of the total population and more than 70\% of the rural population~\parencite{FAO2024}.
    Smallholder farmers form the majority of agricultural producers, yet they remain highly vulnerable to climate variability.

    Kenya’s Arid and Semi-Arid Lands (ASALs) cover more than 80\% of the country, host over 70\% of livestock, and are home to around 36\% of the population~\parencite{IUCN2021,NDMA2021,UNEPDHI2021}.
    Farmers in these regions depend almost entirely on rain-fed crops and livestock, making them extremely vulnerable to droughts and erratic rainfall.

    The drought between 2020 and 2023, the worst in four decades, illustrates the severity of this risk: yields declined by up to 70\%, 2.6 million livestock were lost, and 4.4 million people required urgent food assistance~\parencite{TheStar2024,NDMA2024}.
    These shocks highlight the fragility of rural livelihoods and the absence of effective financial safety nets to protect farmers.

    Our guiding research question is: \emph{How can blockchain-based weather insurance provide affordable, transparent, and automatic protection for rural farmers in Africa who are exposed to increasing climate risks?}

    % 1.1 Problem Analysis
    \subsection{Problem Analysis}\label{subsec:problem-analysis}
    Smallholder farmers in Kenya are uniquely vulnerable to climate risks.
    About 98\% of Kenya’s agricultural systems are rain-fed~\parencite{GoK2017}, while irrigation development is limited due to infrastructural and environmental constraints~\parencite{WairimuND}.
    Traditional coping strategies such as crop diversification, off-farm work, and borrowing are insufficient to withstand the growing severity of climate shocks.
    As a result, households often fall into poverty traps, selling livestock or assets after droughts and struggling to recover in subsequent seasons.
    The recent 2020--2023 drought, the worst in four decades, illustrates the magnitude of the challenge: in some regions, crop yields dropped by up to 70\%, more than 2.6 million livestock died, and 4.4 million people required urgent food assistance~\parencite{OCHA2023,TheStar2024}.
    These shocks do not only affect farmers individually but also ripple across Kenya’s economy, since agriculture contributes over 20\% of national GDP and supports the majority of rural livelihoods~\parencite{WorldBank2024}.

    Conventional agricultural insurance is largely absent in Kenya, and where it does exist, it suffers from deep structural weaknesses.
    High administrative costs make premiums unaffordable for smallholder farmers~\parencite{Dominguez2024}.
    Claims processing is slow and heavily manual, which delays payouts and exacerbates farmers’ financial vulnerability during shocks~\parencite{Chainlink2021}.
    A lack of transparency around pricing and claims fosters widespread mistrust~\parencite{Dominguez2024}.
    Moreover, coverage remains minimal: in Sub-Saharan Africa, fewer than 3\% of farmers are insured, leaving more than 97\% unprotected~\parencite{WorldBank2022}.

    Weather Index Insurance (WII) has emerged as a potential tool to address these challenges, because it relies on measurable weather data such as rainfall or temperature thresholds to trigger payouts.
    This reduces delays, administrative costs, and moral hazard compared to traditional indemnity-based insurance~\parencite{Baagoe2020,Sibiko2018}.
    Studies show that WII adoption can reduce poverty, improve household welfare, and encourage investment in improved seeds and fertilizers.
    Yet despite this potential, uptake remains very limited.
    Farmers often lack awareness or financial literacy, making WII appear too complex~\parencite{Janzen2020}.
    Basis risk remains a major concern, as mismatches between weather station data and on-farm realities can result in payouts that do not reflect actual losses, undermining trust~\parencite{Jensen2016}.
    Affordability is another barrier, since many farmers lack liquidity at the beginning of the planting season, precisely when premiums are due.

    % 1.2 Benefits of Blockchain in Insurance
    \subsection{Benefits of Blockchain in Insurance}\label{subsec:blockchain-benefits}
    Blockchain technology, with its decentralized ledger and automated smart contracts, is particularly well-suited for microinsurance solutions targeting smallholder farmers.
    By eliminating intermediaries, automating claims, and ensuring transparency, blockchain can reduce operational costs, increase trust, and make insurance accessible even in remote areas~\parencite{Dominguez2024,Shetty2022}.
    Moreover, parametric microinsurance, where payouts are triggered by measurable weather events, benefits from blockchain’s immutable and auditable infrastructure, ensuring rapid and verifiable payments. \ldots

    % 2 Business Model and Perspective
    \section{Business Model and Perspective}\label{sec:business-model}
    Smallholder farmers in Kenya, particularly those living in the Arid and Semi-Arid Lands (ASALs), represent the primary target market for decentralized parametric insurance.
    These regions cover more than 80\% of the country’s land area, host over 70\% of the livestock population, and are home to roughly 36\% of the national population~\parencite{IUCN2021,UNEPDHI2021}.
    With more than 98\% of agriculture dependent on rain-fed systems, smallholder farmers are disproportionately exposed to climate variability and shocks~\parencite{GoK2017}.
    Agriculture employs around 40\% of Kenya’s total population and over 70\% of the rural population, yet fewer than 1\% of farmers currently purchase agricultural insurance, leaving the vast majority unprotected~\parencite{FAO2024b,MoA2023}.

    % 2.1 Target Market
    \subsection{Target Market}\label{subsec:target-market}
    The stakeholders in Kenya’s agricultural insurance ecosystem are diverse and interdependent.
    Farmers are the primary end-users.
    Insurers and micro-insurers underwrite and distribute weather-index products, while global reinsurers provide the capital buffers needed to make large-scale coverage feasible~\parencite{Artemis2017,BASIS2017}.
    The Government of Kenya, through its National Agricultural Insurance Policy (NAIP), and regulatory agencies play a critical role in shaping policy and supervising products~\parencite{AfricanClimate2024,MoA2023}.
    International donors and development partners fund pilots, subsidize premiums, and provide technical assistance, as seen in the Kenya Livestock Insurance Program (KLIP)~\parencite{WorldBank2022}.
    Mobile money providers such as Safaricom’s M-Pesa enable efficient premium collection and direct payouts\parencite{Oxford2017}.

    % 2.2 High-Level Concept
    \subsection{High-Level Concept}\label{subsec:high-level-concept}
    \ldots

    % 3 Implementation
    \section{Implementation}\label{sec:implementation}
    Overview of implementation goals, strategy and architecture.

    The goal for this project was providing a showcase prototype .
    Therefore, the focus was on providing the most important features, a neat, easy UI and a good test-setup to showcase the functionality and laying the groundwork for possible extensions.
    These extensions and some of the more advanced features that were not provided in the prototype will be discussed in sections~\ref{subsec:security-risk} and~\ref{subsec:outlook}
    It is important to note that the project is not a complete MVP (minimum viable product), due to the constraints when it comes to funding for example Chainlink as provider for weather data or using actual currency as a payment method.

    % 3.1 Development Process & Strategy
    \subsection{Development Process \& Strategy}\label{subsec:dev-strategy}
    The strategy for the implementation was largely based on a learning process.
    At first the focus was mainly on exploring the basics of Solidity based Smart Contracts and creating an early working prototype.
    Once this was achieved the focus shifted to code quality, finding efficient and well established solutions to tackle current challenges and at the end, integrating all of this into an easy-to-use UI .


    The most important guidelines and ideas for the implementation were:

    \begin{enumerate}[1]
        \item \textbf{Division of Work:} The work was divided mostly into two parts.
        One part was the overarching architecture and product design, including specifically the smart contract development using Solidity as well as researching existing libraries and standards to implement and use.
        The other part, equally important was researching and setting up the tools and frameworks for everything else: The project and testing setup using Hardhat and writing the actual test-code, implementing the UI, using React, Managing the Repository, and bug-fixing.
        This division was important for numerous reasons, but especially because it allowed for mutual verification and simultaneously the possibility to truly focus on specific topics.
        \item \textbf{Testing Setup:} The Testing-Setup within Hardhat and using Node.js made finding bugs and testing after changes much more efficient and also allows for neat and compact showcases of the features and working prototype code.
        An excerpt from this testing setup and what it shows can be seen in figure~\ref{fig:test-set-up}.
        \item \textbf{Research:} Research was an important, continuous process throughout the project, for finding technical solutions but also possible issues with the current state of the project.
        \item \textbf{Usage of AI:} AI was used throughout the project for research purposes, when trying to find fitting libraries, tools and standards, as well as making repetitive tasks, like writing test code more efficient.
    \end{enumerate}

    An important challenge was to align technical progress with the economic modeling.
    Once a working prototype was achieved, communication between these two distinct work streams came into focus.
    Daily synchronization and change management helped in finding ways to implement the most important parts of the business model, whilst simplifying other parts, like using a mock weather oracle, instead of Chainlink as a source of weather data.
    The strategy during the development process was to implement based on an objective hierarchy:

    \begin{enumerate}[1]
        \item Working Smart Contract with deterministic payouts
        \item Investment logic
        \item Investor incentive model (yield / compound interest)
        \item Tests and Code quality
        \item Working frontend for showcases and manual testing
        \item \ldots
    \end{enumerate}

    These are just the most important tasks that are included in the final prototype.
    For further necessary steps towards a finished product refer to~\ref{subsec:outlook}.

    % 3.2 System (Macro) Architecture
    \subsection{System (Macro) Architecture}\label{subsec:system-architecture}
    The macro architecture of the system is centered around the RainyDayFund smart contract, which acts as the core of the decentralized insurance platform.
    It is designed to be extensible for integration with oracles (such as Chainlink), \texttt{ERC20 tokens} as payment and is integrated within a frontend built with React.
    The architecture emphasizes modularity, and ease of use for both farmers and investors, as well as neat and easy testing and showcasing.

    \begin{figure}[H]
        \centering
        \includegraphics[scale=0.3]{graphics/Architectural_Overview}
        \caption{Architectural Overview \\ \textit{Source: Author's own.}}
        \label{fig:architecture}
    \end{figure}

    As shown in figure~\ref{fig:architecture} the users (both farmers and investors) own a crypto wallet and connect this to the frontend.
    The frontend then allows them to easily invest or buy policies for the current season with their crypto funds, by programmatically interacting with the Smart Contract.
    The smart contract then mints a policy token for farmers or deposits the investors funds in the vault, so that they can receive a yield.
    All funds are stored in a Risk-Pool (in our case the ERC4626 vault).
    The farmers can claim their policies at the end of the season and, if all conditions are met, i.e.\ the weather oracle gets and returns weather data, below the threshold and the farmer making the claim actually owns the policy tokens, receive their coverage funds.
    Investors may withdraw their funds, receiving all the accumulated yield.
    The \texttt{Payout Engine} symbolizes the possibility of transferring out funds from the crypto wallet to widely used mobile-money services, like M-Pesa.
    This is already a well established process and not included in the project itself, should however be noted, because it allows the farmers to actually immediately use their payouts in their everyday life.

    The concrete timeline for the process is based on the crop season and has 5 phases:
    \begin{enumerate}
        \item \textbf{Active Phase:} In the active phase policies can be bought and investments can be made.
        The premium for the current season is already set at the beginning through the auction-based system explained in section~\ref{sec:business-model} \textcolor{red}{check wether this links correctly!}.
        No claims or withdrawals can be made during this phase.
        \item \textbf{Inactive Phase:} Now the policies can no longer be bought, but investments can still be made.
        This allows for a speculative market for the insurance policies, while keeping the possibility of introducing more liquidity from capital investors.
        \item \textbf{Claim Phase:} The claim phase starts upon the actual crop season ending.
        Farmers may now try to claim their policies and receive their coverage payouts, while investors have to wait until this phase is over.
        \item \textbf{Withdrawal Phase:} Now investors may choose to withdraw their capital and receive their yield, while farmers wait until the next season is started.
        \item \textbf{Finished Phase:} The season is over, unclaimed policies are now invalid and not-withdrawn investments are used as liquidity for the next season, earning yield on the current investment (including the earned yield), creating a compounding effect.
    \end{enumerate}

    % 3.3 Smart Contract Design (Micro)
    \subsection{Smart Contract Design (Micro)}\label{subsec:smart-contract-design}
    The project is built around a Solidity smart contract using the Ethereum Virtual Machine.
    This central RainyDayFund contract can be called using the front end and holds all the exposed functionality for the consumer.
    Investors use the \texttt{invest()} and \texttt{withdraw()} methods while insurees can buy and claim their policies in batches using the \texttt{buyPolicy()} and \texttt{claimPolicies()} functions.
    The contract leverages the ERC1155 token standard for policy tokens, enabling efficient batch operations and flexible policy management.

    As shown in Figure~\ref{fig:initial-contract-design}, the initial contract made use of the \texttt{1155 token standard} for the \texttt{policyTokens}, to take advantage of the cost-efficient batch operations~\parencite{ERC1155}.
    To help with testing the (Mock)-MUSCD was created, implementing the ERC20 interface, to emulate the behavior of common stablecoins like USDC, which are used as payment and funding for the riskpool.
    This allowed for the free minting and full control over the coins and made testing the functionality of the main contract possible, even within the bounds of Remix VM .
    Therefore, testing was quicker and easier at the beginning, without the need to always deploy on the Sepolia Testnet or an immediate, complete setup, for example with hardhat.

    Initially there were multiple issues:
    \begin{enumerate}[1]
        \item \textbf{Mapping for storing addresses:} Initially we used a mapping to store who had bought policies and was thereby eligible to claim them.
        However, there were some issues with this: Firstly, the policies lost all their value when they were resold, because the address of the new owner would not be tracked in the mapping.
        Moreover, storing all of this data was very gas-inefficient and redundant, because ownership of the policyToken is proof of ownership for the policy.
        \item \textbf{No proper incentive for investors:} We had already planned to give the investors some incentive to keep their funds in the riskpool, even after the season would end, to increase liquidity for the next season.
        The idea to achieve this, was, to use compound interest or create some other form of yield over time.
        However, implementing this with the initial setup proved difficult and created vastly unfair results \textcolor{red}{Hier vllt nochmal genauer erkl\"aren}
        \item \textbf{Weather oracle:} The initial mock-up for the weather oracle was just a number that could be accessed and stored on the contract directly, as an initial mock-up for testing purposes.
        Nevertheless, this was far from the goal of using chainlink as a decentralized oracle to reliably and transparently access the data.
        \item \textbf{Attack vectors:} Lastly there were some attack-vectors, especially for the owner.
        The \texttt{startNewSeason()} function could be used at any point during the season, thus rendering unclaimed policyTokens worthless and not allowing for investment-withdrawals to be made.
    \end{enumerate}

    \begin{figure}[H]
        \centering
        \includegraphics[scale=0.5]{graphics/ClassDiagram_Old}
        \caption{Initial Contract Design \\ \textit{Source: Author's own.}}
        \label{fig:initial-contract-design}
    \end{figure}

    // Solving of issues -> explain new structure

    % 3.4 Tooling & DevOps Pipeline
    \subsection{Tooling \& DevOps Pipeline}\label{subsec:tooling-devops}
    The development process utilized a modern toolchain centered around Hardhat for smart contract development and testing, Node.js for scripting, and React for the frontend.
    Continuous integration and version control were managed via GitHub, with automated testing and linting to ensure code quality.
    The test setup, as shown in Figure~\ref{fig:test-set-up}, highlights the importance of robust testing and code coverage in the project.

    \begin{figure}[H]
        \centering
        \begin{minipage}[b]{0.48\textwidth}
            \centering
            \includegraphics[width=0.95\linewidth]{graphics/Passing_Tests}
            \includegraphics[width=0.95\linewidth]{graphics/Test_Coverage}
            \caption*{(a) Passing Tests and Coverage}
        \end{minipage}\hfill
        \begin{minipage}[b]{0.48\textwidth}
            \centering
            \includegraphics[width=0.95\linewidth]{graphics/Test_Code}
            \caption*{(b) Test Code}
        \end{minipage}
        \caption{Test Setup. (a) Passing tests and coverage; (b) Excerpt of test code. \\ \textit{Source: Author's own.}}
        \label{fig:test-set-up}
    \end{figure}

    % 3.5 Testing & Quality Assurance
    \subsection{Testing \& Quality Assurance}\label{subsec:testing-qa}
    Testing was a critical component of the implementation.
    The project employed unit tests, integration tests, and code coverage analysis to ensure the reliability and correctness of the smart contracts.
    Automated tests were run frequently during development to catch regressions early and to validate new features.
    The use of Hardhat and related plugins facilitated a comprehensive and efficient testing workflow.

    % 3.6 Security & Risk Mitigation
    \subsection{Security \& Risk Mitigation}\label{subsec:security-risk}
    Challenges and risk mitigation were central to the design and implementation of the RainyDayFund contract.
    Key concerns included preventing unauthorized access, ensuring the integrity of the season lifecycle, and protecting against common smart contract vulnerabilities such as reentrancy and integer overflows.
    The contract design incorporated best practices such as checks-effects-interactions, use of OpenZeppelin libraries for access control and safe math, and thorough testing of edge cases.
    Future work will focus on further decentralizing the oracle mechanism and enhancing the robustness of the contract against emerging threats.

    % 3.7 Outlook & Next Steps
    \subsection{Outlook \& Next Steps}\label{subsec:outlook}
    Looking ahead, the project aims to expand its feature set by integrating real-world weather data via decentralized oracles, introducing more sophisticated investor incentive mechanisms, and exploring scalability solutions such as layer-2 networks.
    Additional insurance products and partnerships with local organizations are also envisioned to increase the impact and reach of the platform.

     // Maybe include this:
    For production / next steps:
    \begin{enumerate}[1]
        \item Proper CI setup
        \item Auction Model for price finding
        \item Chainlink integration
        \item Fledged out Frontend
        \item \ldots
    \end{enumerate}


    % 4 Conclusion
    \section{Conclusion}\label{sec:conclusion}
    \ldots

% ---------- Bibliography ----------
    \printbibliography

    \appendix
    \section*{Appendices}\label{sec:appendix}
    Add any supplementary material here.

\end{document}
