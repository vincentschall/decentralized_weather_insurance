\documentclass[11pt,a4paper]{article}

% ---------- Packages ----------
\usepackage[utf8]{inputenc}
\usepackage[T1]{fontenc}
\usepackage{lmodern}          % Better fonts
\usepackage{geometry}         % Page margins
\geometry{margin=1in}
\usepackage{setspace}         % Line spacing
\onehalfspacing
\usepackage{graphicx}         % For figures
\usepackage{hyperref}         % Clickable references
\usepackage{csquotes}         % Required for biblatex
\usepackage[style=authoryear,maxcitenames=2,backend=biber]{biblatex}

% ---------- Bibliography file ----------
\addbibresource{references.bib}

% ---------- Title ----------
\title{\textbf{The Rainy-Day Fund: Decentralized Parametric Insurance for Smallholder Farmers in Kenya}}
\author{
    Group A: Ellena, Sabina, Noah, Vincent\\
    \textit{University of Basel, Blockchain Challenge 25}
}
\date{\today}

\begin{document}
    \maketitle

    \begin{abstract}
        Smallholder farmers in Kenya face serious challenges due to climate variability, as more than 98\% of agriculture depends on rain-fed systems and much of the country’s arable land lies in arid and semi-arid regions (ASALs).
        Frequent droughts, unpredictable rainfall, and extreme weather events have reduced crop yields, threatened food security, and put rural livelihoods at risk.
        Traditional risk mitigation strategies remain insufficient, while irrigation development is limited due to infrastructural and environmental constraints.
        Weather index insurance (WII) has emerged as a practical solution because it provides affordable, quick, and reliable payouts based on objective weather data, while reducing administrative costs and moral hazard.
        Building on this idea, the Rainy-Day Fund introduces a decentralized, blockchain-based parametric insurance system.
        By using smart contracts and mobile payment platforms, it can deliver timely payouts to farmers, strengthen their resilience, and help them reinvest in their farms.
    \end{abstract}

    \textbf{Keywords:} Smallholder farmers; Kenya; climate change; weather index insurance; parametric insurance; blockchain; agricultural resilience; arid and semi-arid lands.

% ---------- Sections ----------
    \section*{Introduction}
    Agriculture is the backbone of Kenya’s economy, contributing approximately 21.3\% to the country’s GDP in 2024~\parencite{WorldBank2024}.
    It is the largest employer in the country, providing livelihoods for over 40\% of the total population and more than 70\% of the rural population~\parencite{FAO2024}.
    Smallholder farmers form the majority of agricultural producers, yet they remain highly vulnerable to climate variability.

    Kenya’s Arid and Semi-Arid Lands (ASALs) cover more than 80\% of the country, host over 70\% of livestock, and are home to around 36\% of the population~\parencite{IUCN2021,NDMA2021,UNEPDHI2021}.
    Farmers in these regions depend almost entirely on rain-fed crops and livestock, making them extremely vulnerable to droughts and erratic rainfall.

    \section*{Problem Analysis}
    Smallholder farmers in Kenya are uniquely vulnerable to climate risks.
    About 98\% of Kenya’s agricultural systems are rain-fed~\parencite{GoK2017}, while irrigation development is limited due to infrastructural and environmental constraints~\parencite{WairimuND}.
    Traditional coping strategies such as crop diversification, off-farm work, and borrowing are insufficient to withstand the growing severity of climate shocks. \ldots

    \section*{Benefits of Blockchain in Insurance}
    Blockchain technology, with its decentralized ledger and automated smart contracts, is particularly well-suited for microinsurance solutions targeting smallholder farmers \parencite{Dominguez2024,Shetty2022}.
    By eliminating intermediaries, automating claims, and ensuring transparency, blockchain can reduce operational costs, increase trust, and make insurance accessible even in remote areas. \ldots

    \section*{Target Market \& Stakeholders}
    \ldots

    \section*{High-Level Concept}
    \ldots

    \section*{Macro Architecture}
    \ldots

    \section*{Implementation Strategy}
    \ldots

    \section*{Micro Architecture / Technical Design}
    \ldots

    \section*{Challenges \& Risk Mitigation}
    \ldots

    \section*{Outlook \& Next Steps}
    \ldots

    \section*{Conclusion}
    \ldots

% ---------- Bibliography ----------
    \printbibliography

    \appendix
    \section*{Appendices}
    Add any supplementary material here.

\end{document}
